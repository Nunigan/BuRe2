\section{Prüfung 18.08.2016}

\subsection{Wortmarke}
gleich wie ''Prüfung 17.08.2017''

\subsection{Urheberrecht}
gleich wie ''Prüfung 17.08.2017''

\subsection{Patentrecht}
\begin{itemize}
	\item In folgenden Ländern sollte man aus \textbf{strategischen Gründen} ein Patent anmelden:
	\begin{itemize}
		\item Im Produktionsland der geschützten Technologie.
		\item Im Verkaufsland der geschützten Technologie.
		\item \textbf{Nicht} (zwingend) im Heimatland des Erfinders.
		\item Am Hauptsitz des Unternehmens.
		\item Um in einem Land Mitbewerber zu blockieren.
	\end{itemize}
	\item Man kann ein Patent auch anmelden, wenn die Technologie erst \textbf{nach dem Anmeldetag veröffentlicht} ist.
	\item Es ist \textbf{nicht} erlaubt, auf Produkte Hinweise bezüglich Patente wie folgt zu schreiben: ''\textbf{weltweit} geschützt; weltweit patentiert, weltumfassender Schutz''
	\item Eine Erfindung kann \textbf{nicht} zum Patent angemeldet werden, wenn der Gegenstand der Erfindungen nicht neu ist, wenn der Gegenstand sich für den Fachmann nicht in naheliegender Weise aus dem Stand der Technik ergibt und wenn die Erfindung nicht gewerblich anwendbar ist.
	\item \textbf{PVÜ} (Pariser Verbandsübereinkunft) ist die Vorstufe zum Europäischen Patentübereinkommen.
	\item Das \textbf{TRIPS} (Agreement on trade-related aspects of Intellectual Property Rights) bezieht sich auf die Vermarktung und den Handel von Rechten am geistigen Eigentum.
	\item Folgende wichtige Informationen kann man einer \textbf{Patentanmeldung mit Recherchenbericht} entnehmen:
	\begin{itemize}
		\item Anmeldetag
		\item Vertragsstaaten
		\item Anmelder
		\item Patentansprüche
		\item Zeichnungen
		\item X,Y Dokumente mit besonderer Bedeutung alleine betrachtet (X) und in Verbindung mit einer anderen Veröffentlichung derselben Kategorie (Y)
	\end{itemize}
	\item \textbf{Software} ist nach dem Europäischen Patentübereinkommen patentierbar in dem eine Erfindung ein PC-Programm enthalten darf, aber der Gegenstand der Erfindung ein technisches Problem lösen muss, z.B. ein neuer erfinderischer Algorithmus.
\end{itemize}

\subsection{Umgang mit Kundendaten}
Online-Apotheke, Angaben bei Kundenkonto: Vor-/Nachname, Adresse, Telefonnummer, E-Mail, Geburtsdatum/-ort, Nationalität, Beruf, med. Fachgebiet und Tätigkeitsbereich. Bei Bestellung: Verwendungszweck der Artikel inkl. Angabe von Krankheit/Unfall sowie Zahlungsinformationen
\begin{itemize}
	\item Die Angabe der Zahlungsinfos bei Bestellung erscheint zum Zweck der Zahlungsabwicklung \textbf{geeignet/erforderlich} und ist deshalb in Übereinstimmung mit dem DSG.
	\item Aus datenschutzrechtlicher Sicht sollte eine Bestellung auch \textbf{ohne Kundenkonto} (als Gast) möglich sein.
	\item Die Pflicht zur Angabe von Krankheit oder Unfall verstösst \textbf{nicht} gegen das \textbf{datenschutzrechtliche Grundprinzip der Datensicherheit}.
	\item Zu Marketingzwecken dürfen die erhobenen Daten grundsätzlich nur verwendet werden, wenn dies in der \textbf{Datenbearbeitungserklärung} oder in \textbf{AGB} aufgeführt wird und der Kunde bei Bestellung dieselben zuvor \textbf{akzeptiert} hat.
\end{itemize}

\subsection{Verwaltung von Kundendaten}
Verwaltung der Kundendatenbank im Ausland. Unter welchen Voraussetzungen erlaubt:
\begin{itemize}
	\item Die Übermittlung von Kundendaten ins Ausland aus Kostengründen ist an \textbf{rechtliche Voraussetzungen} gebunden, auch wenn die tieferen Kosten den Kunden durch Preissenkungen weitergegeben werden.
	\item Das ausländische IT-Unternehmen muss sich der Online-Apotheke vertraglich zur Gewährleistung eines \textbf{angemessenen Schutzes} der zu übermittelnden Daten verpflichten.
	\item Die Übermittlung der Daten ins Ausland ist auch ohne \textbf{explizite Einwilligung} des Kunden möglich, wenn das datenempfangende Unternehmen seinen Sitz in einem Staat hat, der gemäss Einschätzung des EDÖB über eine Gesetzgebung verfügt, die einen \textbf{adäquaten Schutz} der übermittelten Daten gewährleistet.
	\item (In Anschluss an vorherigen Punkt:) Die betroffenen Kunden sollten gemäss dem datenschutzrechtlichen Grundprinzip der Transparenz trotzdem über die Bekanntgabe ihrer Daten an Empfänger im Ausland \textbf{informiert werden}.
\end{itemize}

\subsection{AGB}
\begin{itemize}
	\item Wenn ein Kunde die AGB ausdruckt und korrigiert und diese Version mit einer Produktbestellung schickt und anschliessend das Produkt kommentarlos erhält, kann dieser davon ausgehen, dass die Firma mit seinen \textbf{Änderungen} stillschweigend einverstanden ist.
	\item Ohne eine anderweitige Regelung in AGB stellt der Kunde bei Bestellung im Onlineshop den Antrag zum \textbf{Abschluss eines Kaufvertrages}.
	\item Die AGB der Firma enthalten keine \textbf{Gewährleistungs- und Garantiebestimmungen}, weshalb die gesetzlichen anwendbar sind.
	\item \textbf{Unklare Bestimmungen} in AGB werden im Zweifelsfall \textbf{zugunsten} der Kunden ausgelegt.
\end{itemize}

\subsection{AGB}
AGB verweisen auf Schweizer Recht als das anwendbare und auf Dietlikon/Kt.ZH als Gerichtsstand.
\begin{itemize}
	\item Das Recht eines Kunden, die Firma an seinem eigenen Wohnsitz zu \textbf{belangen}, gilt nicht nur für CH-Kunden. sondern auch für solche im benachbarten \textbf{Ausland}.
	\item Kunden aus der \textbf{EU} oder dem \textbf{EWR} verfügen ggü. der Firma über ein gesetzliches \textbf{Widerrufsrecht}.
	\item Falls Kunden aus Italien ihre Rechnung nicht bezahlen, muss die Firma damit rechnen, dass sie solche Kunden \textbf{in Italien belangen} muss.
	\item In einem allfälligen Gerichtsverfahren gegen jmd. aus Italien muss die Firma damit rechnen, dass das Gericht nach \textbf{italienischem Recht} beurteilen wird.
\end{itemize}

\subsection{Straftatbestände}
\textbf{B} erwirbt von \textbf{H} im Internet eine Liste mit 100 E-Mail-Adressen des Finanzinstituts \textbf{K}. \textbf{H} hat frei einsehbare Mail-Adressen gesammelt. \textbf{B} sendet eine Mail mit einem Hyperlink mit der Aufforderung diesen anzuklicken. Im erscheinenden Online-Formular soll die E-Banking-Vertragsnummer und das Passwort angegeben werden um angebliche Sicherheitslücken zu finden. Im Mail verwendet \textbf{B} das Layout und Logo des Finanzinstituts. Als Absender verwendet er den Namen eines angeblichen Mitarbeiters von \textbf{K}. Zwei Adressaten folgtem dem Link und \textbf{K} loggte sich in deren Konto ein und überwies auf ein mit falschem Ausweis eröffnetes Konto CHF 10'000.--.
\begin{itemize}
	\item \textbf{Unbefugtes Eindringen} in ein Datenverarbeitungssystem (Art. 143bis StGB)
	\item \textbf{Betrügerischer Missbrauch} einer Datenverarbeitungsanlage (Art. 147 StGB)
\end{itemize}